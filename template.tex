% Author: Castamere
\documentclass[UTF8, a4paper, 12pt, AutoFakeBold]{ctexart} % 使用ctex以支持中文字符

\usepackage{titling}
\title{智能知识侦察助手}

% MARK: 字体字号
\usepackage{geometry}
\setmainfont{Times New Roman}
\newcommand{\sanhao}{\fontsize{16pt}{24pt}\selectfont}
\newcommand{\xiaosan}{\fontsize{15pt}{22.5pt}\selectfont}
\newcommand{\sihao}{\fontsize{14pt}{21pt}\selectfont}
\newcommand{\xiaosi}{\fontsize{12pt}{18pt}\selectfont}
\newcommand{\wuhao}{\fontsize{10.5pt}{15.75pt}\selectfont}
\newcommand{\xiaowu}{\fontsize{9pt}{13.5pt}\selectfont}

\renewcommand\CJKrmdefault{SimSun(0)}
\setCJKmainfont{simsun.ttc}[AutoFakeBold]
\geometry{left=3.18cm,right=3.18cm,top=2.54cm,bottom=2.54cm}
\linespread{1.5}

% 引用相关
\RequirePackage{natbib}
\bibliographystyle{references/castamere}
\bibpunct{[}{]}{;}{s}{,}{,} % 第一,二个参数为括号样式,可以换成小括号
\usepackage[hidelinks]{hyperref} % 用于超链接
\newcommand{\fullcite}[1]{(\citeauthor{#1}, \citeyear{#1})\cite{#1}} % 自定义引用指令

\pagestyle{empty} % 去掉页眉页脚等,与引用无关

%%%%%%%%%%% 正文

\begin{document}

\section{期刊}

 [序号] 作者. 题名\nobreak[J]. 刊名, 出版年份, 卷号(期号): 起止页码.

这是引用一篇中文期刊\cite{李静2021多语言}(作者数<3)

这是引用一篇中文期刊\cite{张德海2018基于}(作者数>3)

这是引用一篇外文期刊\cite{Deng2013ApplicationOE}(作者数<3)

这是引用一篇外文期刊\cite{you2022knowledge}(作者数>3)

\section{论文集}

 [序号] 作者. 题名\nobreak[A]. 见(英文用In): 主编. 论文集名\nobreak[C]. 出版地: 出版者, 出版年. 起止页码.

这是引用一篇中文论文集\cite{封晨}

这是引用一篇外文论文集\cite{dong2014knowledge}

\section{学位论文}

 [序号] 作者. 题名\nobreak[D]. 保存地点: 保存单位, 年.

这是引用一篇中文学位论文\cite{姚歆蕾2021基于}

这是引用一篇外文学位论文\cite{w2021e}

\bibliography{references/references} % 使用\bibliography{}在此处列出所有参考文献

\end{document}