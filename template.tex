% Author: Castamere
\documentclass[UTF8, a4paper, 12pt, AutoFakeBold, twoside]{ctexart} % 使用ctex以支持中文字符

\usepackage{titling}
\title{智能知识侦察助手}

% MARK: 字体字号
\usepackage{geometry}
\setmainfont{Times New Roman}
\newcommand{\sanhao}{\fontsize{16pt}{24pt}\selectfont}
\newcommand{\xiaosan}{\fontsize{15pt}{22.5pt}\selectfont}
\newcommand{\sihao}{\fontsize{14pt}{21pt}\selectfont}
\newcommand{\xiaosi}{\fontsize{12pt}{18pt}\selectfont}
\newcommand{\wuhao}{\fontsize{10.5pt}{15.75pt}\selectfont}
\newcommand{\xiaowu}{\fontsize{9pt}{13.5pt}\selectfont}

\renewcommand\CJKrmdefault{SimSun(0)}
\setCJKmainfont{simsun.ttc}[AutoFakeBold]
\geometry{left=3.18cm,right=3.18cm,top=2.54cm,bottom=2.54cm}
\linespread{1.5}

% MARK: 章节命名
\ctexset{
	section={
		name={第,章},
		format=\raggedright\sanhao\heiti,
		beforeskip=0.5ex,
		afterskip=0ex,
		aftername=\enspace,
        number=\arabic{section},
	},
	subsection={
		format=\raggedright\sihao\heiti,
		beforeskip=1ex,
		afterskip=1ex,
		aftername=\enspace,
        number=\arabic{section}.\arabic{subsection},
	},
	subsubsection={
		format=\raggedright\xiaosi\heiti,
		beforeskip=0ex,
		afterskip=0ex,
		aftername=\enspace,
        number=\arabic{section}.\arabic{subsection}.\arabic{subsubsection},
	},
	contentsname={\begin{center}\heiti\sanhao\bfseries{目 \quad 录}\end{center}}
}

% MARK: 目录相关
\usepackage{tocloft}
\setlength\cftbeforetoctitleskip{0ex}
\setlength\cftaftertoctitleskip{0ex}
\setlength\cftbeforepartskip{0ex}
\setlength\cftbeforesecskip{0ex}
\renewcommand{\cftsecfont}{\xiaosi\heiti}
\renewcommand{\cftsubsecfont}{\wuhao\songti}
\renewcommand{\cftsubsubsecfont}{\wuhao\fangsong}

% MARK: 页眉页脚
\usepackage{fancyhdr}

\fancypagestyle{tocstyle}{
	\fancyhf{}
	\renewcommand{\headrulewidth}{0pt}
	\fancyfoot[C]{\xiaowu\thepage}
}

\fancypagestyle{mainstyle}{
	\fancyhf{}
    \setlength{\headheight}{15.75pt}
	\renewcommand{\headrulewidth}{0.3pt}
	\fancyhead[CO]{\wuhao\texttt{浙江理工大学本科毕业设计(论文)}}
	\fancyhead[CE]{\wuhao\texttt{\thetitle}}
	\fancyfoot[CE,CO]{\xiaowu\thepage}
}

% 引用相关
\RequirePackage{natbib}
\bibliographystyle{references/castamere}
\bibpunct{[}{]}{;}{s}{,}{,} % 第一,二个参数为括号样式,可以换成小括号
\usepackage[hidelinks]{hyperref} % 用于超链接
\newcommand{\fullcite}[1]{(\citeauthor{#1}, \citeyear{#1})\cite{#1}} % 自定义引用指令

\usepackage{lipsum}
% MARK: 正文

\begin{document}

% MARK: 摘要
\pagestyle{empty}
\pagenumbering{gobble}

\phantomsection\addcontentsline{toc}{section}{\sihao 摘要}

摘要

\newpage

\phantomsection\addcontentsline{toc}{section}{\sihao Abstruct}

Abstruct

\newpage
% MARK: 目录

\pdfbookmark[1]{目录}{bookmark:目录}
\pagestyle{tocstyle}
\pagenumbering{Roman}
\tableofcontents
\clearpage

% MARK: 图表
\pagestyle{mainstyle}
\pagenumbering{arabic}

\section{图\&表}

\subsection{图}

\lipsum[1-4]

\subsubsection{单图}
\subsubsection{多图}

\subsection{表}
\subsubsection{三线表}
\subsubsection{跨页表}

\section{公式}

\section{参考文献}

\subsection{期刊}

[序号] 作者. 题名\nobreak[J]. 刊名, 出版年份, 卷号(期号): 起止页码.

这是引用一篇中文期刊\cite{李静2021多语言}(作者数<3)

这是引用一篇中文期刊\cite{张德海2018基于}(作者数>3)

这是引用一篇外文期刊\cite{Deng2013ApplicationOE}(作者数<3)

这是引用一篇外文期刊\cite{you2022knowledge}(作者数>3)

\subsection{论文集}

[序号] 作者. 题名\nobreak[A]. 见(英文用In): 主编. 论文集名\nobreak[C]. 出版地: 出版者, 出版年. 起止页码.

这是引用一篇中文论文集\cite{封晨}

这是引用一篇外文论文集\cite{dong2014knowledge}

\subsection{学位论文}

[序号] 作者. 题名\nobreak[D]. 保存地点: 保存单位, 年.

这是引用一篇中文学位论文\cite{姚歆蕾2021基于}

这是引用一篇外文学位论文\cite{w2021e}


% MARK: 参考文献

\phantomsection\addcontentsline{toc}{section}{参考文献}

\bibliography{references/references} % 使用\bibliography{}在此处列出所有参考文献

\end{document}