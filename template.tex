% Author: Castamere
\documentclass[UTF8, a4paper, 12pt, AutoFakeBold, twoside]{ctexart} % 使用ctex以支持中文字符

\usepackage{titling}
\title{智能知识侦察助手}

% MARK: 字体字号
\usepackage{geometry}
\setmainfont{Times New Roman}
\newcommand{\sanhao}{\fontsize{16pt}{24pt}\selectfont}
\newcommand{\xiaosan}{\fontsize{15pt}{22.5pt}\selectfont}
\newcommand{\sihao}{\fontsize{14pt}{21pt}\selectfont}
\newcommand{\xiaosi}{\fontsize{12pt}{18pt}\selectfont}
\newcommand{\wuhao}{\fontsize{10.5pt}{15.75pt}\selectfont}
\newcommand{\xiaowu}{\fontsize{9pt}{13.5pt}\selectfont}

\renewcommand\CJKrmdefault{SimSun(0)}
\setCJKmainfont{simsun.ttc}[AutoFakeBold]
\geometry{left=3.18cm,right=3.18cm,top=2.54cm,bottom=2.54cm}
\linespread{1.5}

% MARK: 章节命名
\ctexset{
	section={
		name={第,章},
		format=\raggedright\sanhao\heiti,
		beforeskip=0.5ex,
		afterskip=0ex,
		aftername=\enspace,
        number=\arabic{section},
	},
	subsection={
		format=\raggedright\sihao\heiti,
		beforeskip=1ex,
		afterskip=1ex,
		aftername=\enspace,
        number=\arabic{section}.\arabic{subsection},
	},
	subsubsection={
		format=\raggedright\xiaosi\heiti,
		beforeskip=0ex,
		afterskip=0ex,
		aftername=\enspace,
        number=\arabic{section}.\arabic{subsection}.\arabic{subsubsection},
	},
	contentsname={\begin{center}\heiti\sanhao\bfseries{目 \quad 录}\end{center}}
}

% MARK: 目录相关
\usepackage{tocloft}
\setlength\cftbeforetoctitleskip{0ex}
\setlength\cftaftertoctitleskip{0ex}
\setlength\cftbeforepartskip{0ex}
\setlength\cftbeforesecskip{0ex}
\renewcommand{\cftsecfont}{\xiaosi\heiti}
\renewcommand{\cftsecleader}{\cftdotfill{\cftdotsep}}
\renewcommand{\cftsubsecfont}{\wuhao\songti}
\renewcommand{\cftsubsubsecfont}{\wuhao\fangsong}

% MARK: 页眉页脚
\usepackage{fancyhdr}

\fancypagestyle{tocstyle}{
	\fancyhf{}
	\renewcommand{\headrulewidth}{0pt}
	\fancyfoot[C]{\xiaowu\thepage}
}

\fancypagestyle{mainstyle}{
	\fancyhf{}
    \setlength{\headheight}{15.75pt}
	\renewcommand{\headrulewidth}{0.3pt}
	\fancyhead[CO]{\wuhao\texttt{浙江理工大学本科毕业设计(论文)}}
	\fancyhead[CE]{\wuhao\texttt{\thetitle}}
	\fancyfoot[CE,CO]{\xiaowu\thepage}
}

% MARK: 引用相关
\RequirePackage{natbib}
\bibliographystyle{references/castamere}
\bibpunct{[}{]}{;}{s}{,}{,} % 第一,二个参数为括号样式,可以换成小括号
\usepackage[hidelinks]{hyperref} % 用于超链接
\newcommand{\fullcite}[1]{(\citeauthor{#1}, \citeyear{#1})\cite{#1}} % 自定义引用指令
\renewcommand{\refname}{\begin{center} \heiti\sanhao 参考文献 \end{center}}

% MARK: 图表公式

% 公式
\usepackage{amsmath}

% 图

% 表
\usepackage{booktabs, multirow, longtable, array, multicol}

% 编号与命名
\usepackage{caption, graphicx, float}
\graphicspath{{figure/}}

\renewcommand{\figurename}{图}
\renewcommand{\tablename}{表}

\numberwithin{equation}{section}
\numberwithin{figure}{section}
\numberwithin{table}{section}

\renewcommand{\theequation}{\thesection-\arabic{equation}}
\renewcommand{\thefigure}{\thesection-\arabic{figure}}
\renewcommand{\thetable}{\thesection-\arabic{table}}

\DeclareCaptionFont{five}{\zihao{5}\bfseries}
\captionsetup[figure]{labelsep=space,font=five}
\captionsetup[table]{labelsep=space,font=five,position=top}

% MARK: 其他

\usepackage{lipsum} % lipsum
\usepackage{pdfpages} % include pdf pages

\begin{document}

% MARK: 前言
\includepdf[pages=-]{preface/诚信声明.pdf}
\pagestyle{empty}
\pagenumbering{gobble}

% \phantomsection\addcontentsline{toc}{section}{\sihao 摘 \quad 要}
\phantomsection
\pdfbookmark[1]{摘要}{bookmark:摘要}
\cftaddtitleline{toc}{part}{\sihao\heiti\mdseries 摘 \quad 要}{}

{\begin{center}\heiti\xiaosan\mdseries 摘 \quad 要\end{center}}

\lipsum[1-3]

\vspace{1ex}
\noindent{\heiti 关键词:}在线学习系统;在线复习系统;遗忘曲线;Next.js;etc.;etc.
\clearpage

% \phantomsection\addcontentsline{toc}{section}{\sihao Abstruct}
\phantomsection
\pdfbookmark[1]{Abstruct}{bookmark:Abstruct}
\cftaddtitleline{toc}{part}{\sihao\mdseries Abstruct}{}

{\begin{center}\xiaosan\mdseries Abstract\end{center}}

\lipsum[4-6]

\vspace{1ex}
\noindent\textbf{Keywords: }Physics; Chemistry; Mathmatics; etc.; etc.
\clearpage
\pdfbookmark[1]{目录}{bookmark:目录}
\pagestyle{tocstyle}
\pagenumbering{Roman}
\tableofcontents
\clearpage

% MARK: 正文
\pagestyle{mainstyle}
\pagenumbering{arabic}

\section{图\&表}

\subsection{图}

\subsubsection{单图}

单行单图如图 \ref{fig:demo} 所示

\begin{figure}[htbp]
    \centering
    \includegraphics[width=0.9\linewidth]{demo.png}
    \caption{demo}
    \label{fig:demo}
\end{figure}

\subsubsection{多图}

单行多图如图 \ref{fig:login}, \ref{fig:reg} 所示

\begin{figure}[H]
    \begin{minipage}[t]{0.49\linewidth}
        \centering
        \includegraphics[width=0.95\columnwidth]{login.png}
        \caption{Login}\label{fig:login}
    \end{minipage}
    \begin{minipage}[t]{0.49\linewidth}
        \centering
        \includegraphics[width=0.95\columnwidth]{reg.png}
        \caption{Register}\label{fig:reg}
    \end{minipage}
\end{figure}

\newpage

\subsection{表}
\subsubsection{三线表}

\begin{table}[h]
    \caption{demo}
    \label{tab:1}
    \xiaowu
    \centering
    \begin{tabular}{c c c c}
        \hline
        \textbf{Name}        & \textbf{Value} \\
        \hline

        ADF Statistic        & -0.501         \\
        p-value              & 0.892          \\
        Critical Values:1\%  & -3.493         \\
        Critical Values:5\%  & -2.892         \\
        Critical Values:10\% & -2.583         \\
        \hline
    \end{tabular}
\end{table}
\subsubsection{跨页表}

\begin{longtable}{>{\sihao}c >{\wuhao }l >{\sihao}c >{\wuhao }l}
    \caption{在线学习标准}                                                                                                                                                                                         \\
    \toprule
    \textbf{\wuhao 标准类型}                                & \multicolumn{1}{c}{\wuhao  \textbf{标准内容}} & \textbf{\wuhao 标准类型}                             & \multicolumn{1}{c}{\wuhao  \textbf{标准内容}} \\
    \midrule
    \endhead

    \bottomrule
    \endfoot
    \bottomrule
    \endlastfoot

    \multirow{10}{*}{\rotatebox[origin=c]{90}{1. 平台}}     & 1. MOOC                                       & \multirow{11}{*}{\rotatebox[origin=c]{90}{3. 评估} } & 20. 电子学习人格化                            \\
                                                            & 2. 移动学习系统                               &                                                      & 21. 电子学习体验                              \\
                                                            & 3. 微软Teams                                  &                                                      & 22. 学生集中程度                              \\
                                                            & 4. MoodleRec                                  &                                                      & 23. 在线教学的有效性                          \\
                                                            & 5. Web 2.0                                    &                                                      & 24. 电子学习的成效                            \\
                                                            & 6. 移动学习平台                               &                                                      & 25. 组织、教学、技术                          \\
                                                            & 7. 移动教学平台                               &                                                      & 26. 感知满意度                                \\
                                                            & 8. 混合型教学平台                             &                                                      & 27. 感知有用性                                \\
                                                            & 9. 在线教学平台                               &                                                      & 28. 学生的看法                                \\
                                                            & 10. 教学管理系统                              &                                                      & 29. 学生准备                                  \\
    \multirow{10}{*}{\rotatebox[origin=c]{90}{2. 评估标准}} & 1. 5维评价模型                                &                                                      & 30. 学习成绩                                  \\
                                                            & 2. Kirkpatrick模型                            & \multirow{8}{*}{\rotatebox[origin=c]{90}{4. 模型}}   & 1.卷积神经网络                                \\
                                                            & 3. 系统实用性量表                             &                                                      & 2. BP 神经网络                                \\
                                                            & 4. Technological Acceptance (TAM)             &                                                      & 3. 互联网 +                                   \\
                                                            & 5. SWOT 分析                                  &                                                      & 4. 自主学习                                   \\
                                                            & 6. 平衡计分卡 (BSC)                           &                                                      & 5. 紧急远程教育                               \\
                                                            & 7. 计划行为理论 (TPB)                         &                                                      & 6. 紧急远程教学                               \\
                                                            & 8. 预期确认模型 (ECM)                         &                                                      & 7. Personal Zed E-learning                    \\
                                                            & 9. 心流理论                                   &                                                      & 8. 紧急远程学习 (ERL)                         \\
                                                            & 10. E-Learning System Model                   & \multirow{5}{*}{\rotatebox[origin=c]{90}{5. 方法}}   & 1. 微课教学方法                               \\
                                                            & 1. 服务质量                                   &                                                      & 2. 翻转课堂                                   \\
                                                            & 2. 学习态度                                   &                                                      & 3. 虚拟教室                                   \\
                                                            & 3. 学习过程                                   &                                                      & 4. 脑电图 (EEG)                               \\
                                                            & 4. 学习效果                                   &                                                      & 5. 心率变化 (HRV)                             \\
                                                            & 5. 学习投入                                   &                                                      & 1. 在线教学不均衡                             \\
                                                            & 6. 用户满意度                                 &                                                      & 2. 缺乏理性自省                               \\
                                                            & 7. 实用性                                     &                                                      & 3. 教育科目的价值被削弱                       \\
    \multirow{12}{*}{\rotatebox[origin=c]{90}{3. 评估}}     & 8. 电子学习的实用性                           & \multirow{8}{*}{\rotatebox[origin=c]{90}{6. 问题}}   & 4. 社会隔离                                   \\
                                                            & 9. 性别                                       &                                                      & 5. 教学环境                                   \\
                                                            & 10. 采用技术                                  &                                                      & 6. 课堂教学                                   \\
                                                            & 11. 教学质量                                  &                                                      & 7. 学生需求                                   \\
                                                            & 12. 学习成绩                                  &                                                      & 8. 使用数字工具的准备情况                     \\
                                                            & 13. 技术使用                                  &                                                      & 9. 缺乏能力                                   \\
                                                            & 14. 学习热情                                  &                                                      & 10. 消极表现                                  \\
                                                            & 15. 学习兴趣                                  &                                                      & 11. 缺乏经验                                  \\
                                                            & 16. 学生和教育工作者的态度                    & \multirow{3}{*}{\rotatebox[origin=c]{90}{7. 趋势}}   & 1. 社交网络的使用                             \\
                                                            & 17. 认知能力发展                              &                                                      & 2. 语义网                                     \\
                                                            & 18. 教育环境                                  &                                                      & 3. 智能技术                                   \\
                                                            & 19. 智力活动                                  &                                                      &

    \label{tab:tech}
\end{longtable}
\section{公式}

\subsection{单行公式}

\emph{The Schrödinger equation} is given by equation \ref{eq:1}.

\begin{equation}
    \label{eq:1}
    i\hbar \frac{\partial}{\partial t} \psi(\mathbf{r}, t) = \hat{H} \psi(\mathbf{r}, t)
\end{equation}

where $\hbar$ is the reduced Planck constant, $\psi(\mathbf{r}, t)$ is the wave function, and $\hat{H}$ is the Hamiltonian operator.

\subsection{跨行公式}

The Schrödinger equation in three dimensions with a time-dependent potential $V(\mathbf{r}, t)$ is given by equation \ref{eq:2}.

\begin{multline}
    \label{eq:2}
    i\hbar \frac{\partial}{\partial t} \Psi(\mathbf{r}, t) = \left[\frac{-\hbar^2}{2m} \nabla^2 + V(\mathbf{r}, t)\right] \Psi(\mathbf{r}, t) \\
    = -\frac{\hbar^2}{2m} \left(\frac{\partial^2}{\partial x^2} + \frac{\partial^2}{\partial y^2} + \frac{\partial^2}{\partial z^2}\right) \Psi(\mathbf{r}, t) + V(\mathbf{r}, t) \Psi(\mathbf{r}, t)
\end{multline}

where $\Psi(\mathbf{r}, t)$ is the wave function, $m$ is the mass of the particle, $\hbar$ is the reduced Planck constant, and $\nabla^2$ is the Laplacian operator.

\subsection{方程组}

\emph{Maxwell's equations} are given by equation \ref{eq:3} through \ref{eq:6}.

\begin{align}
    \label{eq:3} \nabla \cdot \mathbf{E}  & = \frac{\rho}{\varepsilon_0}                                                   \\
    \label{eq:4} \nabla \cdot \mathbf{B}  & = 0                                                                            \\
    \label{eq:5} \nabla \times \mathbf{E} & = -\frac{\partial \mathbf{B}}{\partial t}                                      \\
    \label{eq:6} \nabla \times \mathbf{B} & = \mu_0 \mathbf{J} + \mu_0\varepsilon_0 \frac{\partial \mathbf{E}}{\partial t}
\end{align}

Here, $\mathbf{E}$ and $\mathbf{B}$ represent the electric and magnetic fields respectively, $\rho$ is the charge density, $\mathbf{J}$ is the current density, $\varepsilon_0$ is the permittivity of free space, and $\mu_0$ is the permeability of free space.


\section{参考文献}

\subsection{期刊}

[序号] 作者. 题名\nobreak[J]. 刊名, 出版年份, 卷号(期号): 起止页码.

这是引用一篇中文期刊\cite{李静2021多语言}(作者数<3)

这是引用一篇中文期刊\cite{张德海2018基于}(作者数>3)

这是引用一篇外文期刊\cite{Deng2013ApplicationOE}(作者数<3)

这是引用一篇外文期刊\cite{you2022knowledge}(作者数>3)

\subsection{论文集}

[序号] 作者. 题名\nobreak[A]. 见(英文用In): 主编. 论文集名\nobreak[C]. 出版地: 出版者, 出版年. 起止页码.

这是引用一篇中文论文集\cite{封晨}

这是引用一篇外文论文集\cite{dong2014knowledge}

\subsection{学位论文}

[序号] 作者. 题名\nobreak[D]. 保存地点: 保存单位, 年.

这是引用一篇中文学位论文\cite{姚歆蕾2021基于}

这是引用一篇外文学位论文\cite{w2021e}
\clearpage

% MARK: 附录
\phantomsection\addcontentsline{toc}{section}{参考文献}
\bibliography{references/references}
\newpage
\phantomsection\addcontentsline{toc}{section}{致谢}

\begin{center} \heiti\sanhao 致谢 \end{center}

\lipsum[7-8]

\begin{flushright}
    Castamere

    202x 年 x 月 xx 日
\end{flushright}

\clearpage
\phantomsection\addcontentsline{toc}{section}{附录}

{\noindent\heiti\sihao 附录}

\end{document}